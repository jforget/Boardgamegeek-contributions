% -*- encoding: utf-8 -*-
%
%    Document providing an index of proper names in Wings of the Motherland
%    Copyright (C) 2020 Jean Forget, licensed under Creative Commons Zero
%

\documentclass[a4paper,twocolumn]{article}
\usepackage[T1]{fontenc}
\usepackage[utf8]{luainputenc}
\usepackage{fancyhdr}

\newcommand{\datenum}{\number\year-%
\ifnum\month<10\relax0\fi\number\month-%
\ifnum\day<10\relax0\fi\number\day}

\renewcommand{\rmdefault}[0]{ppl}
\parindent=0mm
\parskip=0pt

\pagestyle{fancy}
\chead{\datenum\hfill Index of proper names in \textit{Wings of the Motherland} \hfill \thepage}
\cfoot{\scriptsize CC0 (Creative Commons Zero) \\ based on \textit{Wings of the Motherland}, copyright \copyright{} 2018 by Clash of Arms Inc \& J.D. Webster}

\directlua{dofile("extract.lua")}
\newenvironment{texte}{\rmfamily}{}

\begin{document}
\section*{Index of persons mentioned in the scenarios}

Scenarios in which a person is active are in upright characters.
Scenarios where a person is just mentioned are in italics.

\vspace{3mm}

\directlua{extract('P','MP')}

\section*{Index of locations and events mentioned in the scenarios}

East Front (or Eastern Front), Germany, Russia and USSR are not
indexed. Some locations such as Kursk are indexed in the associated
event (Battle of Kursk).

\vspace{3mm}

In the context of the soviet army, ``fronts'' may designate an
organization above armies (army groups for Germany) or this may
designate a region where the fighting occurs. I have mentioned a few
fronts when I think they designate a region, I have discarded the
others when I think they designate an command organization.
But I may have made some mistakes.

\vspace{3mm}

As for person names, when a location is directly involved in the
scenario, the scenario code is typeset in upright characters and when
the location is just mentioned, the code is typeset in italics.
In some cases, it is difficult to determine if a location is directly
involved in the scenario or if it is just mentioned. For example, when
a scenario involves a mission ``en route to location X'', I have
added the scenario code to the location X entry, but in some cases I have
typeset the scenario code in upright characters, in other cases I have
typeset it in italics.

\vspace{3mm}

\directlua{extract('L','ML')}

\section*{Units}

Nation-wide units such as Luftwaffe or the Soviet Army are not indexed.

\vspace{1mm}

\directlua{extract('U','MU')}

\vspace{1mm}

\section*{Aircraft Types}

The index includes the scenario variations and the encounter
tables for mission scenarios, with a ``v'' or ``e''
suffix. Aircraft types are indexed in the variant/encounter parts only if they do not appear in the
scenario main description.

\vspace{1mm}

\directlua{extract('A','MA')}

\section*{Ships}

This index includes both named ships and ship classes.

\vspace{1mm}

% Not "S" for "ship", because "S" is for "scenario". Instead, "B" for "Boat".
\directlua{extract('B','MB')}

\section*{Errata}

Here are a few errata from the scenarios. Not something that can ruin a game, just
a few typos.

\vspace{3mm}

First, Yakovlev, not Yakevlov: various pages of the scenario booklet, not limited
to the scenarios from page 16 to page 114.

\vspace{3mm}

\directlua{extract('E','-')}

\end{document}

=encoding utf-8

=head1 Usage

  lualatex index-WotM-v2.tex

Then use your favorite PDF viewer to display index-WotM-v1.pdf

=head1 Description

This  document  gives  an  index  of proper  names  mentioned  in  the
scenarios  of  I<Wings of  the  Motherland>.  These proper  names  are
separated in two categories. The first category is persons. The second
category  mixes locations  (e.g. Kurksk)  and events  (e.g. Battle  of
Kursk).

This  document includes  also  a  few errata  I  have  spotted in  the
scenario booklet.

=head1 Copyright and License

Copyright (c) 2020, 2023 Jean Forget, licensed under CC0 (Creative Commons Zero)

The  generated  PDF document  is  based  on  the scenario  booklet  of
\textit{Wings of  the Motherland}, copyright  (c) 2018, Clash  of Arms
Inc & J.D. Webster

