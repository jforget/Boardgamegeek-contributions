% -*- encoding: utf-8 -*-
%
%    Document providing an index of proper names in Whistling Death
%    Copyright (C) 2024 Jean Forget, licensed under Creative Commons Zero
%

\documentclass[a4paper,twocolumn]{article}
\usepackage[T1]{fontenc}
\usepackage[utf8]{luainputenc}
\usepackage{fancyhdr}
\usepackage{hanging}

\newcommand{\datenum}{\number\year-%
\ifnum\month<10\relax0\fi\number\month-%
\ifnum\day<10\relax0\fi\number\day}

\renewcommand{\rmdefault}[0]{ppl}
\parindent=0mm
\parskip=0pt

\pagestyle{fancy}
\chead{\datenum\hfill Index of proper names in \textit{Whistling Death} \hfill \thepage}
\cfoot{\scriptsize CC0 (Creative Commons Zero) \\ based on \textit{Whistling Death}, copyright \copyright{} 2003 by Clash of Arms Inc \& J.D. Webster}

\directlua{dofile("extract.lua")}
\newenvironment{texte}{\rmfamily}{}

\begin{document}

Scenarios in which a person, location, aircraft type, unit or ship is
directly referenced are in upright characters. Scenarios are in
italics if the person, location, etc is just mentioned.

\sloppy
\section*{Persons mentioned in the scenarios}

\fussy
\begin{hangparas}{4mm}{1}
\directlua{extract('P','MP')}
\end{hangparas}

\section*{Locations and events mentioned in the scenarios}

Oceans, nations such as Japan or USA are not indexed, but smaller
nations as Philippines are indexed. Some locations such as Guadalcanal
are indexed in the associated event (Guadalcanal, Battle of).

\vspace{2mm}

When a location is directly involved in the scenario, the scenario
code is typeset in upright characters and when the location is just
mentioned, the code is typeset in italics. In some cases, it is
difficult to determine if a location is directly involved in the
scenario or if it is just mentioned. For example, when a scenario
involves a mission ``en route to location X'', I have added the
scenario code to the location X entry, but in some cases I have
typeset the scenario code in upright characters, in other cases I have
typeset it in italics.

\vspace{2mm}

\begin{hangparas}{4mm}{1}
\directlua{extract('L','ML')}
\end{hangparas}

\section*{Units}

Nation-wide units such as Marine Corps or the Japanese Navy are not indexed.

\vspace{2mm}

\begin{hangparas}{4mm}{1}
\directlua{extract('U','MU')}
\end{hangparas}

\section*{Aircraft Types}

The index includes the scenario variations and the encounter
tables for mission scenarios, with a ``v'' or ``e''
suffix. Aircraft types are indexed in the variant/encounter parts only if they do not appear in the
scenario main description.

\vspace{2mm}

\begin{hangparas}{4mm}{1}
\directlua{extract('A','MA')}
\end{hangparas}

\section*{Ships}

This index includes both named ships and ship classes.

\vspace{2mm}

% Not "S" for "ship", because "S" is for "scenario". Instead, "B" for "Boat".
\begin{hangparas}{4mm}{1}
\directlua{extract('B','MB')}
\end{hangparas}

\section*{Errata}

Here are a few errata from the scenarios. Not something that can ruin a game, just
a few typos.

\vspace{3mm}

\begin{hangparas}{4mm}{1}
\directlua{extract('E','-')}
\end{hangparas}

\end{document}

=encoding utf-8

=head1 Usage

  lualatex index-WD-v1.tex

Then use your favorite PDF viewer to display index-WD-v1.pdf

=head1 Description

This  document  gives  an  index  of proper  names  mentioned  in  the
scenarios  of  I<Whistling Death>.  These proper  names  are
separated in two categories. The first category is persons. The second
category  mixes locations  (e.g. Guadalcanal)  and events  (e.g. Battle  of
Guadalcanal).

This  document includes  also  a  few errata  I  have  spotted in  the
scenario booklet.

=head1 Copyright and License

Copyright (c) 2024 Jean Forget, licensed under CC0 (Creative Commons Zero)

The  generated  PDF document  is  based  on  the scenario  booklet  of
\textit{Whistling Death}, copyright  (c) 2003, Clash  of Arms
Inc & J.D. Webster

