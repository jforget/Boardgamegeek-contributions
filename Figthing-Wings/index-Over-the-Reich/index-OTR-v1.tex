% -*- encoding: utf-8 -*-
%
%    Document providing an index of proper names in Over the Reich
%    Copyright (C) 2023 Jean Forget, licensed under Creative Commons Zero
%

\documentclass[a4paper,twocolumn]{article}
\usepackage[T1]{fontenc}
\usepackage[utf8]{luainputenc}
\usepackage{fancyhdr}

\newcommand{\datenum}{\number\year-%
\ifnum\month<10\relax0\fi\number\month-%
\ifnum\day<10\relax0\fi\number\day}

\renewcommand{\rmdefault}[0]{ppl}
\parindent=0mm
\parskip=0pt

\pagestyle{fancy}
\chead{\datenum\hfill Index of proper names in \textit{Over the Reich} \hfill \thepage}
\cfoot{\scriptsize CC0 (Creative Commons Zero) \\ based on \textit{Over the Reich}, copyright \copyright{} 1993 by J.D. Webster \& Clash of Arms}

\directlua{dofile("extract.lua")}
\newenvironment{texte}{\rmfamily}{}

\begin{document}

Scenarios are identified by the page number, a dash and a
sequential number to distinguish scenarios in the same page. If
a scenario is printed on several pages, the first page is used for
the scenario code. For example, \textit{Marauder Disaster} is printed
on pages 50 and 51, therefore its code is 50-4.

When a person, location, etc is directly involved in a scenario,
the scenario code is typeset in upright characters. If the scenario
includes just an indirect mention of this person or location,
its code is typeset in italics.

\section*{Index of persons}

\vspace{3mm}

\directlua{extract('P','MP')}

\section*{Index of locations and events}

Western European Nations are not indexed.

\vspace{3mm}

As for person names, when a location is directly involved in the
scenario, the scenario code is typeset in upright characters and when
the location is just mentioned, the code is typeset in italics.
In some cases, it is difficult to determine if a location is directly
involved in the scenario or if it is just mentioned. For example, when
a scenario involves a mission ``en route to location X'', I have
added the scenario code to the location X entry, but in some cases I have
typeset the scenario code in upright characters, in other cases I have
typeset it in italics.

\vspace{3mm}

\directlua{extract('L','ML')}

\section*{Units}

\vspace{3mm}

\directlua{extract('U','MU')}

\section*{Aircraft Types}

The index includes  neither the scenario variations  nor the encounter
tables for mission scenarios.

\vspace{3mm}

\directlua{extract('A','MA')}

\section*{Errata}

Here are a few errata from the scenarios. Not something that can ruin a game, just
a few typos.

\vspace{3mm}

\directlua{extract('E','-')}

\end{document}

=encoding utf-8

=head1 Usage

  lualatex index-OTR-v1.tex

Then use your favorite PDF viewer to display index-OTR-v1.pdf

=head1 Description

This  document  gives  an  index  of proper  names  mentioned  in  the
scenarios of  I<Over the Reich>.  These proper names are  separated in
two categories.  The first  category is  persons. The  second category
mixes locations (e.g.  Kurksk) and events (e.g. Battle  of Kursk). The
third category includes  units and the fourth  category lists aircraft
types.

This  document includes  also  a  few errata  I  have  spotted in  the
scenario booklet.

=head1 Copyright and License

Copyright (c) 2023 Jean Forget, licensed under CC0 (Creative Commons Zero)

The generated PDF document is based on the booklet of \textit{Over the
Reich}, copyright (c) 1993, J.D. Webster & Clash of Arms

