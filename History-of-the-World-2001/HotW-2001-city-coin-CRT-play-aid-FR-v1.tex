% -*- encoding: utf-8; indent-tabs-mode: nil -*-
%
%     Script LATEX + METAPOST pour construire la version française de l'aide de jeu de History of the World
%     Copyright (C) 2020 Jean Forget
%
%     Voir la licence dans la documentation à la fuin de ce fichier
%
\documentclass[a4paper,twocolumn]{article}
\usepackage[english,frenchb,italian]{babel}
\usepackage[T1]{fontenc}
\usepackage[utf8]{luainputenc}
\usepackage{luamplib}
\usepackage{fancyhdr}

% Voir LATEX guide pratique page 235
% marge haute 25.4 - 12 = 13.4 mm
\topmargin=-12mm
\headheight=10mm
\headsep=4mm
% marge basse 297 - (13.4 + 267 + 4) = 12.6mm
\textheight=240mm

% marge gauche 25,4mm,
% marge droite 210 - (25,4 + 173) = 11,6mm
\oddsidemargin=0mm
\marginparsep=0mm
\textwidth=173mm
\columnsep=8mm

%&Latex

\renewcommand{\rmdefault}[0]{ppl}
\renewcommand{\sfdefault}[0]{phv}
\renewcommand{\ttdefault}[0]{pcr}

\newcommand{\datenum}{\number\year-%
\ifnum\month<10\relax0\fi\number\month-%
\ifnum\day<10\relax0\fi\number\day}

\parindent=0mm
\parskip=0pt

\newenvironment{texte}{\rmfamily\footnotesize}{}

\newcommand{\eps}[0]{$\varepsilon$}
\newcommand{\carte}[1]{``\textit{#1}''}

%\pagestyle{myheadings}
%\markright{\sffamily\upshape\kern 8mm\datenum\hfill {\bfseries\normalsize History of the World}\kern 15mm\hfill Page }
\pagestyle{fancy}
\chead{\datenum\hfill \textit{History of the World} aide de jeu \hfill }
\cfoot{\scriptsize CC0 (Creative Commons Zero), basé sur \textit{History of the World}, \copyright{} 2001, Avalon Hill Games, \copyright{} 2001 Hasbro}

%\newcommand{\image}[2]{\includegraphics[#1]{#2}}
\NoAutoSpaceBeforeFDP

\begin{document}
\begin{texte}
\tolerance=1000

\section{Capitale, cité et monument}

\begin{mplibcode}
beginfig(1);
  xvide =  0; yvide = 40;
  xcite = 60; ycite = 20;
  xcap  =  0; ycap  =  0;
  largeur = 20; hauteur = 5;
  label(btex vide etex, (xvide, yvide));
  label(btex cit\'e etex, (xcite, ycite));
  label(btex capitale etex, (xcap, ycap));
  draw  (xvide - largeur, yvide - hauteur) -- (xvide + largeur, yvide - hauteur)
                                           -- (xvide + largeur, yvide + hauteur)
                                           -- (xvide - largeur, yvide + hauteur)
                                           -- (xvide - largeur, yvide - hauteur);
  draw  (xcite - largeur, ycite - hauteur) -- (xcite + largeur, ycite - hauteur)
                                           -- (xcite + largeur, ycite + hauteur)
                                           -- (xcite - largeur, ycite + hauteur)
                                           -- (xcite - largeur, ycite - hauteur);
  draw  (xcap  - largeur, ycap  - hauteur) -- (xcap  + largeur, ycap  - hauteur)
                                           -- (xcap  + largeur, ycap  + hauteur)
                                           -- (xcap  - largeur, ycap  + hauteur)
                                           -- (xcap  - largeur, ycap  - hauteur);

  (x1, y1) = ((xvide + xcap) / 2 - 40, (yvide + ycap) / 2);
  drawarrow (xvide - largeur, yvide) .. (x1, y1) .. (xcap - largeur, ycap);
  label.lft(btex Placement initial etex, (x1, y1));

  (x2, y2) = ((xvide + 2 * xcite) / 3 - largeur, (yvide + 2 * ycite) / 3);
  drawarrow (xvide, yvide - hauteur) .. (x2, y2) .. (xcite - largeur, ycite);
  label.llft(btex Carte \it Kingdom etex, (x2, y2));

  (x3, y3) = ((xvide + xcite) / 2 + largeur, (yvide + ycite) / 2);
  drawarrow (xcite, ycite + hauteur) .. (x3, y3) .. (xvide + largeur, yvide);
  label.urt(btex Pillage, carte \it Disaster etex, (x3, y3));

  (x4, y4) = ((xcite + xcap ) / 2 + largeur, (ycite + ycap - hauteur ) / 2);
  drawarrow (xcap + largeur, ycap) .. (x4, y4) .. (xcite, ycite - hauteur);
  label.lrt(btex Pillage, carte \it Disaster etex, (x4, y4));

endfig;
\end{mplibcode}

\subsection{Capitale}

      Obtenue lors du placement initial d'un peuple.

      Supprimée lors d'un pillage, remplacée alors par une cité. Idem
      avec une carte \carte{Disaster}.

      Effet : 2 points.

\subsection{Cité}

      Obtenue lorsque l'on pille une capitale. Ou bien créée par certaines cartes
      (royaumes).

      Supprimée lors d'un nouveau pillage ou avec une carte \carte{Disaster}.

      Effet : 1 point.

\subsection{Monument}

      Obtenu lors de la phase de construction, si l'empire contrôle deux
      ressources.

      Supprimé : habituellement, il dure jusqu'à la fin du jeu. Parfois,
      supprimé par une carte \carte{Disaster}.

      Effet : 1 point

      Priorité pour le placement :

\begin{enumerate}
\item La capitale
\item Une cité
\item Une ressource
\end{enumerate}

\section{Monnaie}

\begin{mplibcode}
beginfig(1);
  xrien     =   0; yrien     =  80;
  xflotte   = -60; yflotte   =  50;
  xmonnaie  =   0; ymonnaie  =   0;
  xarmee    = -30; yarmee    = -40;
  xfort     =  90; yfort     = -40;
  xpool     =   0; ypool     = -80;
  largeur = 20; hauteur = 5;
  label(btex rien         etex, (xrien, yrien));
  label(btex flotte       etex, (xflotte, yflotte));
  label(btex monnaie      etex, (xmonnaie, ymonnaie));
  label(btex arm\'ee      etex, (xarmee, yarmee));
  label(btex fort         etex, (xfort, yfort));
  label(btex pool initial etex, (xpool, ypool));
  label(btex d'arm\'ees   etex, (xpool, ypool - 2 * hauteur));
  draw  (xrien - largeur, yrien - hauteur) -- (xrien + largeur, yrien - hauteur)
                                           -- (xrien + largeur, yrien + hauteur)
                                           -- (xrien - largeur, yrien + hauteur)
                                           -- (xrien - largeur, yrien - hauteur);
  draw  (xflotte - largeur, yflotte - hauteur) -- (xflotte + largeur, yflotte - hauteur)
                                               -- (xflotte + largeur, yflotte + hauteur)
                                               -- (xflotte - largeur, yflotte + hauteur)
                                               -- (xflotte - largeur, yflotte - hauteur);
  draw  (xmonnaie  - largeur, ymonnaie  - hauteur) -- (xmonnaie + largeur, ymonnaie - hauteur)
                                                   -- (xmonnaie + largeur, ymonnaie + hauteur)
                                                   -- (xmonnaie - largeur, ymonnaie + hauteur)
                                                   -- (xmonnaie - largeur, ymonnaie - hauteur);
  draw  (xarmee  - largeur, yarmee  - hauteur) -- (xarmee + largeur, yarmee - hauteur)
                                               -- (xarmee + largeur, yarmee + hauteur)
                                               -- (xarmee - largeur, yarmee + hauteur)
                                               -- (xarmee - largeur, yarmee - hauteur);
  draw  (xfort  - largeur, yfort  - hauteur) -- (xfort + largeur, yfort - hauteur)
                                             -- (xfort + largeur, yfort + hauteur)
                                             -- (xfort - largeur, yfort + hauteur)
                                             -- (xfort - largeur, yfort - hauteur);
  draw  (xpool  - largeur, ypool  - 3 * hauteur) -- (xpool + largeur, ypool - 3 * hauteur)
                                                 -- (xpool + largeur, ypool +     hauteur)
                                                 -- (xpool - largeur, ypool +     hauteur)
                                                 -- (xpool - largeur, ypool - 3 * hauteur);

  (x1, y1) = ((xrien - largeur + xflotte) / 2 - 10, (yrien + yflotte + hauteur) / 2);
  drawarrow (xrien - largeur, yrien) .. (x1, y1) .. (xflotte, yflotte + hauteur);
  label.lft(btex Placement initial   etex, (x1, y1 + 2 * hauteur));
  label.lft(btex Carte \it Astronomy etex, (x1, y1));

  (x2, y2) = ((xflotte + xmonnaie - largeur) / 2 - 10, (yflotte + ymonnaie + hauteur) / 2);
  drawarrow (xflotte, yflotte - hauteur) .. (x2, y2) .. (xmonnaie - largeur, ymonnaie + 2);
  label.lft(btex Carte \it Reallocation etex, (x2, y2));

  (x3, y3) = ((xmonnaie + xrien) / 2 - 15, (ymonnaie + yrien) / 2);
  drawarrow (xmonnaie -5, ymonnaie + hauteur) .. (x3, y3) .. (xrien -5, yrien - hauteur);
  label.lft(btex     Fin etex, (x3, y3 + hauteur));
  label.lft(btex du tour etex, (x3, y3 - hauteur));

  (x4, y4) = ((xmonnaie + xarmee) / 2 - 15, (ymonnaie + yarmee) / 2);
  drawarrow (xmonnaie - largeur, ymonnaie - 2) .. (x4, y4) .. (xarmee, yarmee + hauteur);
  label.lft(btex    Rachat d'une etex, (x4, y4 + hauteur));
  label.lft(btex arm\'ee vaincue etex, (x4, y4 - hauteur));

  (x5, y5) = ((xmonnaie + xfort) / 2 - 20, (ymonnaie + yfort) / 2);
  drawarrow (xmonnaie, ymonnaie - hauteur) .. (x5, y5) .. (xfort - largeur, yfort);
  label.lft(btex Achat etex, (x5, y5));

  (x6, y6) = ((xmonnaie + xrien) / 2 + 15, (ymonnaie + yrien) / 2);
  drawarrow (xrien + 5, yrien - hauteur) .. (x6, y6) .. (xmonnaie + 5, ymonnaie + hauteur);
  label.rt(btex Cartes \it Allies  etex, (x6, y6 + 2 * hauteur));
  label.rt(btex \it Pop. explosion etex, (x6, y6));
  label.rt(btex \it Civil service  etex, (x6, y6 - 2 * hauteur));

  (x71, y71) = ((xrien + largeur + xfort) / 2 + 15, (3 * yrien + yfort + hauteur) / 4);
  (x72, y72) = ((xrien + largeur + 2 * xfort) / 3 + 15, (yrien + 3 * (yfort + hauteur)) / 4);
  drawarrow (xrien + largeur, yrien - 2) .. (x71, y71) .. (x72, y72) .. (xfort - 2, yfort + hauteur);
  label.lft(btex           Carte etex, (x72, y72));
  label.lft(btex \it Engineering etex, (x72, y72 - 2 * hauteur));

  (x81, y81) = ((xrien + largeur + 2 * xfort) / 3 + 25, (yrien + 3 * (yfort + hauteur)) / 4);
  (x82, y82) = ((xrien + largeur + xfort) / 2 + 25, (3 * yrien + yfort + hauteur) / 4);
  drawarrow (xfort + 2, yfort + hauteur) .. (x81, y81) .. (x82, y82) .. (xrien + largeur, yrien + 2);
  label.rt(btex D\'efaite       etex, (x82, y82 + 10 * hauteur));
  label.rt(btex Cartes          etex, (x82, y82 +  8 * hauteur));
  label.rt(btex \it Famine      etex, (x82, y82 +  6 * hauteur));
  label.rt(btex \it Plague      etex, (x82, y82 +  4 * hauteur));
  label.rt(btex \it Black Death etex, (x82, y82 +  2 * hauteur));
  label.rt(btex \it Pestilence  etex, (x82, y82));

  (x9, y9) = ((xpool + largeur + xfort - hauteur) / 2 + 15, (ypool + yfort - hauteur) / 2);
  drawarrow (xpool + largeur, ypool - hauteur) .. (x9, y9) .. (xfort, yfort - hauteur);
  label.lrt(btex Conversion etex, (x9, y9));

endfig;
\end{mplibcode}

\subsection{Monnaie}

      Obtenue :
      \begin{itemize}
      \item carte \carte{Reallocation}, permettant de convertir des flottes en pièces de monnaie,
      \item cartes \carte{Allies} et \carte{Population Explosion} donnant deux pièces,
      \item carte \carte{Civil Service} donnant un nombre variable de pièces.
      \end{itemize}

      Effet : rachat d'une armée vaincue, ou bien construction d'un fort

      Perdue à la fin du tour si pas utilisée.

\subsection{Fort}

      Obtenu :
      \begin{itemize}
      \item en dépensant une pièce de monnaie,
      \item en convertissant une armée qui n'est pas encore placée,
      \item carte \carte{Engineering} qui fournit 2 forts si l'on a une capitale.
      \end{itemize}

      Effet : le fort donne +1 au dé lors des combats en défense et il
      absorbe la première perte.

      Supprimé lorsqu'un combat est perdu ou par les cartes \carte{Famine},
      \carte{Plague}, \carte{Black Death} et \carte{Pestilence}.

\subsection{Flotte}

      Obtenue :
      \begin{itemize}
      \item au déploiement d'un empire, comme indiqué sur la carte de l'empire,
      \item carte \carte{Astronomy}.
      \end{itemize}

      Supprimée : à la fin du tour de l'empire, ou bien convertie en
      monnaie avec une carte \carte{Reallocation}.

      Effet : permet l'expansion à travers les mers et les océans.

\section{Table de combat}

\begin{tabular}[t]{|p{30mm}|p{16mm}|p{10mm}|p{10mm}|}
\hline
       Carte de         & Terrain du          &  Dés de              & Dés du  \\
       l'attaquant      & défenseur           &  l'attaquant         & défenseur  \\
\hline
      (rien)            & normal              &         2D6          &       1D6         \\
\hline
      (rien)            & Grande muraille     &                      &       2D6         \\
\hline
      (rien)            & montagnes           &                      &       2D6         \\
\hline
Expert Troops: Mountains& montagnes           &                      &       1D6         \\
\hline
      (rien)            & forêt               &                      &       2D6         \\
\hline
 Expert Troops: Forests & forêt               &                      &       1D6         \\
\hline
      (rien)            & détroit             &                      &       2D6         \\
\hline
 Expert Troops: Straits & détroit             &                      &       1D6         \\
\hline
      (rien)            & débarquement        &                      &       3D6         \\
\hline
Naval Power             & débarquement        &                      &       2D6         \\
\hline
      (rien)            & fort                &                      & \textit{n}D6+1    \\
\hline
    Siegecraft          & capitale, cité ou
                                fort          &        nD6+1         &                   \\
\hline
Leader (jusqu'au premier
résultat triple)        &                     &         3D6          &                   \\
\hline
     weaponry           &                     & \textit{n}D6+1       &                   \\
\hline
 Jihad (jusqu'à la      &                     &       3D6+\eps       &                   \\
 première défaite)      &                     &                      &                   \\
\hline
 Jihad (après la        &                     &       2D6+\eps       &                   \\
 première défaite)      &                     &                      &                   \\
\hline
 Jihad (après la        &                     &         2D6          &                   \\
 deuxième défaite)      &                     &                      &                   \\
\hline
 Elite (jusqu'à la      &                     & \textit{n}D6+\eps    &                   \\
 première défaite)      &                     &                      &                   \\
\hline


 \end{tabular}

\eps{} : un genre de point de dé fractionnaire, dont la seule utilité est qu'en cas d'égalité de points, il fait basculer la balance en la faveur du joueur.

\textit{n}D6+... : se reporter à une autre ligne pour connaître la valeur de \textit{n}.

\end{texte}
\end{document}

=encoding utf-8

=head1 Description

Script pour construire un fichier  PDF en français contenant les aides
de jeu pour I<History of the World>.


=head1 Utilisation

  lualatex HotW-2001-city-coin-CRT-play-aid-FR-v1.tex

=head1 Description

Ce  fichier  LuaLATEX  donne  le  résumé de  quelques  règles  du  jeu
I<History of  the World> (version  2001). publié par Avalon  Hill puis
par Hasbro.

Les  règles  concernant  les  capitales   et  les  cités,  les  règles
concernant l'utilisation de la monnaie et les règles décrivant les cas
particuliers des combats  sont dispersées dans le livret  de règles et
dans  les cartes  d'événement. Le  but  de cette  aide de  jeu est  de
regrouper ces règles dans des paragraphes courts et simples à lire.

=head1 Copyright et Licence

=head2 Fichier PDF

Copyright (c) 2020 Jean Forget

Le fichier PDF  résultant de ce programme est soumis  à la licence CC0
(Creative Commons Zero).

=head2 Program

Copyright (c) 2020 Jean Forget

Ce programme est diffusé sous la  licence GNU Public License version 1
ou toute version ultérieure.

Voici le texte officiel en anglais du résumé de la GPL :

This program is  free software; you can redistribute  it and/or modify
it under the  terms of the GNU General Public  License as published by
the Free  Software Foundation; either  version 1, or (at  your option)
any later version.

This program  is distributed in the  hope that it will  be useful, but
WITHOUT   ANY  WARRANTY;   without  even   the  implied   warranty  of
MERCHANTABILITY  or FITNESS  FOR  A PARTICULAR  PURPOSE.  See the  GNU
General Public License for more details.

You should  have received  a copy  of the  GNU General  Public License
along  with  this  program;  if   not,  write  to  the  Free  Software
Foundation, Inc., L<https://www.fsf.org/>.

